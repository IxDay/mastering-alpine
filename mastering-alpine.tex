\documentclass[aspectratio=169]{beamer}
\usetheme{Perso}

\usepackage[outputdir=build]{minted}
\setminted{breaklines}
\newminted{text}{frame=single}
\usemintedstyle{tango}
\graphicspath{ {images/} }

\title{Mastering Alpine Linux}
\date{\today}
\author{Maxime Vidori}

\begin{document}

\begin{frame}
  \titlepage
\end{frame}


\begin{frame}{Alpine?}{Never heard of it...}
  What's in the box?
  \begin{itemize}
    \item \textbf{provide a package manager and a small footprint} (v3.5 \char`~4MB).
    \item Based on busybox and musl-libc.
    \item Can be used as a distribution and come with a grsec kernel.
  \end{itemize}
  How can this help me?
  \begin{itemize}
    \item Easier to understand and deploy.
    \item Force you to invest time in your system, and production
      environment.
    \item Reduce security risks by mastering your toolchain,
      \textbf{no more third party unknown containers!}
  \end{itemize}
\end{frame}

\begin{frame}{Alpine?}{musl libc}
  \begin{block}{musl}
    \textit{lightweight, fast, simple, free,} and strives to be
    \textit{correct} in the sense of standards-conformance and safety.
    \end{block}
  \begin{itemize}
    \item Replacement for the \textbf{glibc}, works most of the time.
    \item \textbf{\char`~600KB} vs \textbf{\char`~8MB} for complete .so set.
    \item Some softwares will not compile (I am looking at you \textbf{systemd}).
  \item You can still install it, but this is sketchy and not
      recommended outside a chroot (see the documentation).
  \end{itemize}
\end{frame}

\begin{frame}{Alpine?}{busybox}
  \begin{block}{busybox}
    The Swiss Army knife of Embedded Linux
  \end{block}
  \begin{itemize}
    \item Simple binary with minimal versions of common UNIX utilities
      (rm, ls, ...).
    \item Minimal size (\textbf{\char`~2MB})
    \item Primarily designed as a recovery tool.
    \item Used by major projects such as Debian for the installation.
  \end{itemize}
\end{frame}

\begin{frame}{Alpine?}{apk}
  \textbf{apk} is the tool used to install, upgrade, or delete software on a running alpine system.

  \begin{itemize}
    \item introduce some other dependencies: libcrypto, libssl, libz.
    \item more convenient than a basic scratch image.
    \item good tooling and documentation.
  \end{itemize}
\end{frame}

\begin{frame}{When not to use it}
  \begin{itemize}
    \item The use of \textit{musl-libc} as the core library can cause some
      dependencies to not build.
    \item When building big images the small footprint is no longer an advantage
      (cross compiler can be really huge).
    \item Package library is not exhaustive (10GB big, for main and community),
      this is not a debian distribution,
      if a lot of dependencies are involved do not use it.
  \end{itemize}
\end{frame}

\begin{frame}[fragile]{Tooling}{tips and tricks!}
  \begin{itemize}
    \item virtual package, remove dev dependencies easily\\
      (golang containers can have size reduced from \textbf{\char`~200MB} to
      \textbf{\char`~11MB})
    \item create local mirror for packages (only \char`~5GB for the main repo),
      allow rapid offline builds, push your custom packages.
    \item custom packages Alpine package description file is based on Gentoo Linux
      \textbf{ebuilds}, an easy way to package is to check Archlinux AUR for examples.
  \end{itemize}
\end{frame}


\begin{frame}[fragile]{Tooling}{tips and tricks!}
  \begin{itemize}
\item Start your mirror server\\
      \mintinline{shell}{docker run -p "8080:80" demo/server}
    \item Build your image \\ \mintinline{text}{docker build -t demo/base:1.0 .}
      \begin{minted}{docker}
FROM alpine:3.5
RUN echo "http://bridge.ip:8080" > /etc/apk/repositories
ADD repo-key.rsa.pub /etc/apk/keys
      \end{minted}

    \item Use your pipeline!
      \begin{minted}{docker}
FROM demo/base:1.0
RUN apk add --no-cache -t dependencies hello ...
      \end{minted}

  \end{itemize}
  \flushright{Just use \textbf{minikube} or \textbf{compose}!}

\end{frame}

\begin{frame}{Demo!}
  \LARGE \textbf{... What could possibly go wrong?}
\end{frame}

\begin{frame}{Conclusion}
  \begin{itemize}
    \item When building containers, think size, think network, think build time.
    \item Be sure to use the right tools for the right thing.
    \item Before starting new tools, look at what already exists.
  \end{itemize}
\end{frame}

\begin{frame}{That's all folks}
  \LARGE \textbf{Questions?}

  \begin{flushright}
    \normalsize \color{Green}IxDay\color{Grey}/mastering-alpine
   \end{flushright}
\end{frame}


\end{document}
